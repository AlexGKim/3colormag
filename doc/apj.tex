\documentclass{aastex61}   	% use "amsart" instead of "article" for AMSLaTeX format
\usepackage{geometry}                		% See geometry.pdf to learn the layout options. There are lots.
\geometry{letterpaper}                   		% ... or a4paper or a5paper or ... 
\usepackage{graphicx}				% Use pdf, png, jpg, or eps§ with pdflatex; use eps in DVI mode
\usepackage{amsmath}
\usepackage{amssymb}
\usepackage{natbib}
\usepackage{lineno}
\usepackage{color}
\linenumbers

\begin{document}

\title{Standardization of Type~Ia Supernova Absolute Magnitudes}
\author[0000-0001-6315-8743]{A.~G.~Kim}
\affiliation{    Physics Division, Lawrence Berkeley National Laboratory, 
    1 Cyclotron Road, Berkeley, CA, 94720}
%    
%
%\author{     G.~Aldering}
%\affiliation{    Physics Division, Lawrence Berkeley National Laboratory, 
%    1 Cyclotron Road, Berkeley, CA, 94720}
%
%\author{     P.~Antilogus}
%\affiliation{    Laboratoire de Physique Nucl\'eaire et des Hautes \'Energies,
%    Universit\'e Pierre et Marie Curie Paris 6, Universit\'e Paris Diderot Paris 7, CNRS-IN2P3, 
%    4 place Jussieu, 75252 Paris Cedex 05, France}
%    
%\author{     S.~Bailey}
%\affiliation{    Physics Division, Lawrence Berkeley National Laboratory, 
%    1 Cyclotron Road, Berkeley, CA, 94720}
%
%\author{     C.~Baltay}
%\affiliation{    Department of Physics, Yale University, 
%    New Haven, CT, 06250-8121}
%
%\author{     K.~Barbary}
%\affiliation{
%    Department of Physics, University of California Berkeley,
%    366 LeConte Hall MC 7300, Berkeley, CA, 94720-7300}
%
%\author{    D.~Baugh}
%\affiliation{   Tsinghua Center for Astrophysics, Tsinghua University, Beijing 100084, China }
%
%\author{     K.~Boone}
%\affiliation{    Physics Division, Lawrence Berkeley National Laboratory, 
%    1 Cyclotron Road, Berkeley, CA, 94720}
%\affiliation{
%    Department of Physics, University of California Berkeley,
%    366 LeConte Hall MC 7300, Berkeley, CA, 94720-7300}
%
%\author{     S.~Bongard}
%\affiliation{    Laboratoire de Physique Nucl\'eaire et des Hautes \'Energies,
%    Universit\'e Pierre et Marie Curie Paris 6, Universit\'e Paris Diderot Paris 7, CNRS-IN2P3, 
%    4 place Jussieu, 75252 Paris Cedex 05, France}
%
%\author{     C.~Buton}
%\affiliation{    Universit\'e de Lyon, F-69622, Lyon, France ; Universit\'e de Lyon 1, Villeurbanne ; 
%    CNRS/IN2P3, Institut de Physique Nucl\'eaire de Lyon}
%    
%\author{     J.~Chen}
%\affiliation{   Tsinghua Center for Astrophysics, Tsinghua University, Beijing 100084, China }
%
%\author{     N.~Chotard}
%\affiliation{    Universit\'e de Lyon, F-69622, Lyon, France ; Universit\'e de Lyon 1, Villeurbanne ; 
%    CNRS/IN2P3, Institut de Physique Nucl\'eaire de Lyon}
%    
%\author{     Y.~Copin}
%\affiliation{    Universit\'e de Lyon, F-69622, Lyon, France ; Universit\'e de Lyon 1, Villeurbanne ; 
%    CNRS/IN2P3, Institut de Physique Nucl\'eaire de Lyon}
%
%\author{ S.~Dixon}
%\affiliation{
%    Department of Physics, University of California Berkeley,
%    366 LeConte Hall MC 7300, Berkeley, CA, 94720-7300}
%
%\author{     P.~Fagrelius}
%\affiliation{    Physics Division, Lawrence Berkeley National Laboratory, 
%    1 Cyclotron Road, Berkeley, CA, 94720}
%\affiliation{
%    Department of Physics, University of California Berkeley,
%    366 LeConte Hall MC 7300, Berkeley, CA, 94720-7300}
%
%\author{     H.~K.~Fakhouri}
%\affiliation{    Physics Division, Lawrence Berkeley National Laboratory, 
%    1 Cyclotron Road, Berkeley, CA, 94720}
%  \affiliation{
%    Department of Physics, University of California Berkeley,
%    366 LeConte Hall MC 7300, Berkeley, CA, 94720-7300}
%
%\author{     U.~Feindt}
%\affiliation{The Oskar Klein Centre, Department of Physics, AlbaNova, Stockholm University, SE-106 91 Stockholm, Sweden}
%
%\author{     D.~Fouchez}
%\affiliation{    Centre de Physique des Particules de Marseille, 
%    Aix-Marseille Universit\'e , CNRS/IN2P3, 
%    163 avenue de Luminy - Case 902 - 13288 Marseille Cedex 09, France}
%    
%\author{     E.~Gangler}  
%\affiliation{    Clermont Universit\'e, Universit\'e Blaise Pascal, CNRS/IN2P3, Laboratoire de Physique Corpusculaire,
%    BP 10448, F-63000 Clermont-Ferrand, France}
%    
%\author{     B.~Hayden}
%\affiliation{    Physics Division, Lawrence Berkeley National Laboratory, 
%    1 Cyclotron Road, Berkeley, CA, 94720}
%
%\author{     W.~Hillebrandt}
%\affiliation{    Max-Planck-Institut f\"ur Astrophysik, Karl-Schwarzschild-Str. 1,
%D-85748 Garching, Germany}
%
%\author{     M.~Kowalski}
%\affiliation{    Institut fur Physik,  Humboldt-Universitat zu Berlin,
%    Newtonstr. 15, 12489 Berlin}
%\affiliation{ DESY, D-15735 Zeuthen, Germany}
%
%\author{     P.-F.~Leget}
%\affiliation{    Clermont Universit\'e, Universit\'e Blaise Pascal, CNRS/IN2P3, Laboratoire de Physique Corpusculaire,
%    BP 10448, F-63000 Clermont-Ferrand, France}
%    
%\author{     S.~Lombardo}
%\affiliation{    Institut fur Physik,  Humboldt-Universitat zu Berlin,
%    Newtonstr. 15, 12489 Berlin}
%    
%\author{     J.~Nordin}
%\affiliation{    Institut fur Physik,  Humboldt-Universitat zu Berlin,
%    Newtonstr. 15, 12489 Berlin}
%    
%\author{     R.~Pain}
%\affiliation{    Laboratoire de Physique Nucl\'eaire et des Hautes \'Energies,
%    Universit\'e Pierre et Marie Curie Paris 6, Universit\'e Paris Diderot Paris 7, CNRS-IN2P3, 
%    4 place Jussieu, 75252 Paris Cedex 05, France}
%     
%\author{     E.~Pecontal}
%\affiliation{   Centre de Recherche Astronomique de Lyon, Universit\'e Lyon 1,
%    9 Avenue Charles Andr\'e, 69561 Saint Genis Laval Cedex, France}
%    
%\author{    R.~Pereira}
% \affiliation{    Universit\'e de Lyon, F-69622, Lyon, France ; Universit\'e de Lyon 1, Villeurbanne ; 
%    CNRS/IN2P3, Institut de Physique Nucl\'eaire de Lyon}
% 
% \author{    S.~Perlmutter}
% \affiliation{    Physics Division, Lawrence Berkeley National Laboratory, 
%    1 Cyclotron Road, Berkeley, CA, 94720} 
%\affiliation{
%    Department of Physics, University of California Berkeley,
%    366 LeConte Hall MC 7300, Berkeley, CA, 94720-7300}
%    
% \author{    D.~Rabinowitz}
% \affiliation{    Department of Physics, Yale University, 
%    New Haven, CT, 06250-8121}
%    
% \author{    M.~Rigault} 
%\affiliation{    Institut fur Physik,  Humboldt-Universitat zu Berlin,
%    Newtonstr. 15, 12489 Berlin}
%     
% \author{    D.~Rubin}
% \affiliation{    Physics Division, Lawrence Berkeley National Laboratory, 
%    1 Cyclotron Road, Berkeley, CA, 94720}
%    \affiliation{   Space Telescope Science Institute, 3700 San Martin Drive, Baltimore, MD 21218}
% 
% \author{    K.~Runge}
% \affiliation{    Physics Division, Lawrence Berkeley National Laboratory, 
%    1 Cyclotron Road, Berkeley, CA, 94720}
% 
% \author{    C.~Saunders}
% \affiliation{    Physics Division, Lawrence Berkeley National Laboratory, 
%    1 Cyclotron Road, Berkeley, CA, 94720}
%
%\author{    C.~Sofiatti}
%\affiliation{    Physics Division, Lawrence Berkeley National Laboratory, 
%    1 Cyclotron Road, Berkeley, CA, 94720} 
%\affiliation{
%    Department of Physics, University of California Berkeley,
%    366 LeConte Hall MC 7300, Berkeley, CA, 94720-7300}
%
%\author{    N.~Suzuki}
%\affiliation{    Physics Division, Lawrence Berkeley National Laboratory, 
%    1 Cyclotron Road, Berkeley, CA, 94720}
%
%\author{     S.~Taubenberger}
%\affiliation{    Max-Planck-Institut f\"ur Astrophysik, Karl-Schwarzschild-Str. 1,
%D-85748 Garching, Germany}
%
%\author{     C.~Tao}
%\affiliation{   Tsinghua Center for Astrophysics, Tsinghua University, Beijing 100084, China }
%\affiliation{    Centre de Physique des Particules de Marseille, 
%    Aix-Marseille Universit\'e , CNRS/IN2P3, 
%    163 avenue de Luminy - Case 902 - 13288 Marseille Cedex 09, France}
%   
%\author{     R.~C.~Thomas}
%\affiliation{    Computational Cosmology Center, Computational Research Division, Lawrence Berkeley National Laboratory, 
%    1 Cyclotron Road MS 50B-4206, Berkeley, CA, 94720}
%    
%\collaboration{(The Nearby Supernova Factory)}


%\begin{abstract}
%\end{abstract}
%
%\keywords{supernovae: general; supernovae}

\section{Introduction}
The objective of this work is to determine the relative absolute magnitudes of SNe~Ia.  This is done in two steps.
The first step is to model peak-magnitude colors as a function of parameters.  The second step is to model the relative achromatic magnitudes
as a function of the same parameters.
The two models in combination give the  relative absolute magnitudes of SNe~Ia.

The following toy example illustrates this two-step approach.  Suppose we were standardizing the absolute magnitudes of a set of spherical
blackbodies.
In the first step, an arbitrarily normalized Planck's law parameterized by temperature would describe supernova colors.  The second step, the
modeling of achromatic magnitudes as a function of temperature, would describe the radii as a function of temperature.
If the blackbodies all had the same radius, the arbitrarily normalized Planck's law would also describe supernova absolute magnitudes.
If the distribution of radii were temperature-independent, then again the
arbitrarily normalized Planck's law would also describe supernova absolute magnitudes but with random dispersion that directly mirrors the
radius distribution.
If the distribution of radii were temperature-dependent, then a model for the temperature-dependent achromatic magnitude together
with Planck's law would  describe supernova absolute magnitudes.


The color analysis is expounded upon in the earlier draft article.  This note focuses on the relative achromatic magnitudes
and the resulting relative absolute-magnitude residuals.

\section{Color Analysis Summary}
\label{color:sec}
The color analysis is described in a separate article.
The color analysis gives per-supernova parameters
$\vec{\Delta}$,  $\overrightarrow{EW}_{Ca}$,  $\overrightarrow{EW}_{Si}$,  $\vec{\lambda}_{Si}$,   $\vec{x}_1$,  $\vec{A}_{V,p}$,
where each vector has $N$ elements, one for each of the supernovae in the sample.  
Relative magnitude offsets are given by $\vec{\Delta}$; spectral features by $\overrightarrow{EW}_{Ca}$,  $\overrightarrow{EW}_{Si}$,  $\vec{\lambda}_{Si}$;
SALT2 $x_1$ by $\vec{x}_1$; a newly identified color parameter by $\vec{A}_{V,p}$.
The parameters $\vec{g}_0$ and $\vec{g}_1$ also introduced in the color analysis are attributed to dust
and are not included as per-supernova parameters.  CMB-frame redshifts of the supernovae are $\vec{z}$.

In the color analysis, the parameters $\vec{\Delta}$,  $\overrightarrow{EW}_{Ca}$,  $\overrightarrow{EW}_{Si}$,  $\vec{\lambda}_{Si}$,   $\vec{x}_1$,  $\vec{A}_{V,p}$ are not independent by construction.  
It is for this reason that we analyze
relative absolute magnitudes, which are independent.
We define a new set of vectors such that
$\vec{\Delta}_{.0} = (\vec{\Delta} - \Delta[0])[1:]$ with length $N-1$; here each per-supernova parameter is measured relative to the
reference supernova arbitrarily selected to be the one with index 0.\footnote{Python numpy notation is used for vector indexing.}
The data vector for the current analysis is the concatenation of these vectors
\[
\vec{y} = \left\{\vec{\Delta}_{.0} , \overrightarrow{EW}_{Ca\,.0} ,  \overrightarrow{EW}_{Si\,.0},  \vec{\lambda}_{Si\,.0},   \vec{x}_{1\,.0}, \vec{A}_{V,p\,.0}\right\},
\]
which has length $6(N-1)$.  The data covariance $C_{y}$, positive definite,  is given by the color analysis. This 
components of $\vec{y}$ are shown in Fig.~\ref{input:fig}.

\begin{figure}[htbp] %  figure placement: here, top, bottom, or page
   \centering
   \includegraphics[width=3.75in]{../input.pdf}
   \caption{Relative supernova features as a function of relative magnitude offsets that serve as input in the analysis. 
   \label{input:fig}}
\end{figure}

\begin{sloppypar}
The standard deviations of $\vec{y}$ excepting the first element is denoted as
${\vec{\sigma}_{y_{.0}}=\{27\text{\AA},  6.9\text{\AA},  20\text{\AA},   0.98,   0.029 \text{ mag}\}}$.
The  standard deviation of $\vec{A}_{V,p\,.0}$, 0.029,  is comparable to the per element uncertainties from the diagonal of $C$.
All the other features have per-object measurement uncertainties significantly smaller than the population standard deviation.
\end{sloppypar}

\begin{figure}[htbp] %  figure placement: here, top, bottom, or page
   \centering
   \includegraphics[width=2.95in]{../sigma_p.pdf}
   \caption{Black vertical line: Standard deviation of the observed $\vec{A}_{V,p\,.0}$, which has contributions from  measurement
   uncertainty and intrinsic dispersion.
Green: Histogram of the measurement uncertainties in $\vec{A}_{V,p\,.0}$.  
Blue: (Unnormalized) posterior of the $\vec{A}_{V,p\,.0}$ intrinsic dispersion $\sigma_{{A}_{V,p\,.0}}$ .
   \label{sigma_p:fig}}
\end{figure}

\section{Achromatic Magnitude Analysis}
\label{magnitude:sec}
\subsection{Model}
\label{achromaticmodel:sec}
In our model SNe~Ia have features drawn from an intrinsic distribution.  
Those features are related linearly related to achromatic absolute magnitudes with residual random intrinsic dispersion.
The following few paragraphs go into more detail.

The model describes the distribution of the SN-feature parameters relative to those of the fiducial supernova.  The distribution is modeled as a
5-dimensional Normal distribution
centered at $\langle \overrightarrow{\text{feature}}\rangle$ and covariance
matrix
$(\vec{\sigma}_{\text{feature}}I)^T C_{\text{feature}}  (\vec{\sigma}_{\text{feature}}I)$.  The covariance matrix is composed from
its standard deviations $\vec{\sigma}_{\text{feature}}$,
and its correlation matrix $C_{\text{feature}}$.
The feature population is this described by $20$ parameters.  These parameters are given in Table~\ref{population:tab}.
\begin{table}
\begin{center}
\caption{Parameters for the feature population. \label{population:tab}}
\begin{tabular}{rccc}
\hline
Parameter & pdf & Description & \# parameters\\ \hline
$\langle \overrightarrow{\text{feature}}\rangle$ & flat  &  mean & 5\\
$\vec{\sigma}_{\text{feature}}$ & $\text{Cauchy}(\vec{0},4\vec{\sigma}_{y_{.0}})$; $>0$  &  s.d. & 5\\
$C_{\text{feature}}$ & $\text{LKJ}(4)$ &  correlation & 10\\
\hline
\end{tabular}
\end{center}
\end{table}

The model parameterizes the intrinsic relative parameters of the $N-1$ supernovae as shown in Table~\ref{feature:tab}.
They are denoted as $\vec{\theta}_{\mathrm{feature}}$.
The SN indexed by $i$ is associated with the parameters
$\vec{\theta}^{(i)}=\left\{\theta_{EW_{Ca}}[i], \theta_{EW_{Si}}[i], \theta_{\lambda_{Si}}[i], \theta_{x_1}[i], \theta_{p}[i] \right\}$
that are drawn from
\begin{equation}
\vec{\theta}^{(i)} \sim \mathcal{N}(\vec{\theta}_{\mathit{mn}}, (\vec{\theta}_{\sigma}I) C_{\text{feature}}  (\vec{\theta}_{\sigma}I).
\label{feature:pdf}
\end{equation}
These are $5(N-1)$ feature parameters. These parameters are given in Table~\ref{feature:tab}.

\begin{table}
\begin{center}
\caption{Parameters for the supernova features.\label{feature:tab}}
\begin{tabular}{rccc}
\hline
Parameter & pdf & Description & \# parameters\\ \hline
$\vec{\theta}_{EW_{Ca}}$ & Eq.~\ref{feature:pdf} & ${EW}_{Ca}$ features & $N-1$ \\
$\vec{\theta}_{EW_{Si}}$ & Eq.~\ref{feature:pdf}& ${EW}_{Si}$ features & $N-1$ \\
$\vec{\theta}_{\lambda_{Si}}$ &Eq.~\ref{feature:pdf}& ${\lambda}_{Si}$ features & $N-1$ \\
$\vec{\theta}_{x_1}$ & Eq.~\ref{feature:pdf}& ${x}_{1}$ features & $N-1$ \\
$\vec{\theta}_{p}$ & Eq.~\ref{feature:pdf}& $\sigma_p\phi_{\hat{V}}  {p} $ features & $N-1$ \\
\hline
\end{tabular}
\end{center}
\end{table}

The achromatic magnitude is related linearly to the intrinsic features via
cofactors $\vec{\alpha}$ with intrinsic dispersion $\sigma_M$.
With five features there should be five linear coefficients $\vec{\alpha}$.  The quality of the input data is insufficient to
simultaneously constrain $\vec{\theta}_p$ and their associated cofactor, so we do not attempt to make a achromatic magnitude
correction with this feature.
These are 5 parameters for the cofactors and intrinsic dispersion.  These parameters are given in Table~\ref{mag:tab}.  

\begin{table}
\begin{center}
\caption{Parameters for the linear coefficients and residual dispersion.\label{mag:tab}}
\begin{tabular}{rccc}
\hline
Parameter & pdf & Description & \# parameters\\ \hline
$\vec{\alpha}$ & Flat& feature cofactors & 4 \\
$\sigma_M$ & $\text{Cauchy}(0, 0.1)$; $>0$ & intrinsic dispersion & 1\\
\hline
\end{tabular}
\end{center}
\end{table}

For expository and computational convenience, parameters for per-supernova peculiar velocity and
intrinsic magnitude offset are introduced.
These parameters are not uniquely constrained but manifest dispersion due to peculiar velocities and intrinsic magnitude dispersion.
These parameters are given in Table~\ref{dispersion:tab}.  
\begin{table}
\begin{center}
\caption{Parameters for per-supernova magnitude dispersion.\label{dispersion:tab}}
\begin{tabular}{rccc}
\hline
Parameter & pdf & Description & \# parameters\\ \hline
$\vec{v}$ & $\mathcal{N}(0,10^{-3})$ &  peculiar velocity & N \\\
$ \overrightarrow{dm}$ &  $\mathcal{N}(0,\sigma_M)$  &  peculiar magnitude & N \\
$\overrightarrow{\delta z}$ & $\sqrt{(1+\vec{v})/ (1-\vec{v})}-1$ & peculiar redshift& N \\
\hline
\end{tabular}
\end{center}
\end{table}

The parameters $\vec{\mu}$  correspond to the model expectation of the observables and are derived from the parameters previously introduced.
The  prediction for the  magnitude residual relative to reference SN~0, $ \mu[i]; 0 \le i < N-1$,
is modeled as a linear function of the supernova features plus terms for peculiar velocity and intrinsic dispersion.
The remaining elements of $\vec{\mu}$ are the model feature parameters.
These parameters are given in Table~\ref{mu:tab}.  
The model likelihood is described by
\begin{equation*}
\vec{y} \sim \mathcal{N}\left(\vec{\mu}, C_{y}\right).
\end{equation*}

\begin{table}
\begin{center}
\caption{Derived parameters that correspond to the data.\label{mu:tab}}
\begin{tabular}{rccc}
\hline
Parameter & Definition & Description\\ \hline
$ \mu[i]; 0 \le i < N-1$ & $\sum_{k=0}^{3} \alpha[k] \theta_k[i] + \frac{5}{\log{10}}\left(\frac{\delta z[i+1]}{z[i+1]} -\frac{\delta z[0]}{z[0]}\right) + \left(dm[i+1]-dm[0]\right)$ & residual from reference\\
$ \mu[N:6(N-1)]$ & $\left\{\vec{\theta}_{EW_{Ca}},\vec{\theta}_{EW_{Si}}, \vec{\theta}_{\lambda_{Si}}, \vec{\theta}_{x_1},\vec{\theta}_{p} \right\}$ &  features \\
\hline
\end{tabular}
\end{center}
\end{table}



The model does not include redshift measurement uncertainty.  Some of the SNf supernovae have redshift uncertainties of 0.01.  Interpreting
this uncertainty as Gaussian gives peculiar velocities that have singular magnitude corrections.

\subsection{Results}
\label{achromaticresults:sec}
Posterior contours for pairs of the distribution parameters
$\langle \overrightarrow{\text{feature}}\rangle$ and
$\vec{\sigma}_{\text{feature}}$ are shown in Fig.~\ref{confidence:fig}. 

\begin{figure}[htbp] %  figure placement: here, top, bottom, or page
   \centering
   \includegraphics[width=2.95in]{../population.pdf}
   \includegraphics[width=2.95in]{../top.pdf}
   \caption{Left: 
Posterior contours for pairs of
$\langle \overrightarrow{\text{feature}}\rangle$ and
$\vec{\sigma}_{\text{feature}}$.  Right: Posterior contours for pairs of $\vec{\alpha}$ and
$\sigma_M$.  Dashed lines correspond to zero.
The contours correspond to  1, 2, and  3$\sigma$ levels, which in 2-dimensions are
39\%,  86\%, and 98.9\%   confidence levels.
N.B. The package used to make the plot renormalizes ranges in a way that doesn't appear in the plot, need
to figure  this out.
   \label{confidence:fig}}
\end{figure}

As noted in
\S\ref{color:sec}, the measurement uncertainty of ${\vec{A}_{V,p\,.0}}$ is comparable to the observed
population standard deviation.  As a  consequence
the resulting
posterior for  $\sigma_{{A_{V,p\,.0}}}$  accommodates zero intrinsic dispersion, as
seen in both Figs.~\ref{sigma_p:fig} and \ref{confidence:fig}.  If 
a linear coefficient relating ${A_{V,p\,.0}}$
and achromatic magnitude were  fit along with the underlying distribution, the intrinsic  $\vec{\theta}_{A_{V,p\,.0}}$, weakly constrained by the 
low signal-to-noise measurements, would be
free to take on values that eliminate intrinsic dispersion.  As the data does not provide a strong constraint on an underlying Normal distribution,
we do not attempt to fit an $\alpha$ term that corresponds to  ${A_{V,p\,.0}}$.


Posterior contours for pairs  of $\vec{\alpha}$ and
$\sigma_M$, the parameters that describe the achromatic magnitude correction and residual dispersion, are shown in Fig.~\ref{confidence:fig}.
The $1\sigma$ credible intervals are given in Table~\ref{magci:tab}.

\begin{table}
    \centering
    \caption{$1\sigma$ credible intervals for the model parameters that describe the achromatic magnitude.}
    \label{magci:tab}
    \begin{tabular}{ccccc}
        \hline
		 $\alpha_{EW_{Ca}}$ & $\alpha_{EW_{Si}}$ & $\alpha_{\lambda_{Si}}$ & $\alpha_{x_1}$ & $\sigma_M$ \\ 
		\hline
$\left( -11.8^{+6.1}_{-5.0} \right) \times 10^{-4}$ & $\left( 9.0^{+4.7}_{-6.6} \right) \times 10^{-3}$ & $\left( -20.5^{+7.1}_{-6.8} \right) \times 10^{-4}$ & $0.100^{+0.032}_{-0.051}$ & $\left( 93.7^{+8.0}_{-7.9} \right) \times 10^{-3}$ \\ 
		\hline
    \end{tabular}
\end{table}

\subsection{Out of Sample Magnitude Dispersion}
Absolute magnitude standardization performed using one set of supernovae may be applied to another disjoint set.
The dispersion in predicted magnitudes of out-of-sample supernovae is performed using jackknife resampling, in which 
standardization is performed $N-1$ times.  For each standardization one supernova is omitted from the training
and the absolute
magnitude residual after correction of that supernova is determined. 

Not finished with this yet.  A preliminary plot is Fig.~\ref{jk:fig}.

\begin{figure}[htbp] %  figure placement: here, top, bottom, or page
   \centering
   \includegraphics[width=2.95in]{../jkresiduals.png}
   \caption{Absolute magnitude residuals after correction for each supernova where it has been withheld from training. 
   Total uncertainty includes both measurement uncertainty and contributions from peculiar velocity and intrinsic
   magnitude dispersion.\label{jk:fig}}
\end{figure}

\section{Magnitude Residuals and Host-Galaxy Mass}

A systematic difference in the SALT-determined absolute magnitudes of supernovae from low- and high-mass host galaxies is seen in
our sample and many others.  In this section we replicate the analysis of \S\ref{magnitude:sec} while adding 
host masses.

\subsection{Data}

%The output of the color analysis
%described in \S\ref{color:sec} includes the grey magnitude offsets relative to a fiducial supernova $\vec{\Delta}_{.0}$.
%The first $N-1$ elements of $\mu$, described in Table~\ref{mu:tab}, are the model predictions for those offsets.
%Our magnitude residuals after achromatic magnitude correction are $\vec{\Delta}_{.0} - \mu[0:N]$.
%We consider  a model in which the residuals are dependent
%on host-galaxy mass.
%
%For the analysis in this section, the input includes $\vec{\Delta}_{.0} - \mu[0:N]$ from
%the results of both the color analysis of \S\ref{color:sec} and the achromatic magnitude analysis of 
%\S\ref{magnitude:sec}.  The
%uncertainty, $C$, is approximated as the quadratic sum of the covariances of the two analyses.  
%While there is covariance between $\vec{\Delta}_{.0}$ and $\mu[0:N]$ that should be taken into account when
%subtracting the two,  it is not readily available from the analysis
%and its contribution is small relative to the total uncertainties in $\mu[0:N]$.

The data are those described in \S\ref{color:sec} with the addition
host-galaxy masses (in units of $\log{(M_{\mathrm{host}}/M_{\odot})}$) and uncertainties, $\vec{M}_{\text{host}}$  and
$\vec{\sigma}_{M_{\text{host}}}$,
from \citet{2013ApJ...770..108C}. 
These inputs are shown in Fig.~\ref{mass:fig}.

\begin{figure}[htbp] %  figure placement: here, top, bottom, or page
   \centering
   \includegraphics[width=2.95in]{../mass.pdf}
   \caption{Magnitude-offset residuals after achromatic correction versus host-galaxy mass.   Overplotted are the mean
   and $\pm1 \sigma$ mass step.     \label{mass:fig}}
\end{figure}

\subsection{Model}

The achromatic model of \S\ref{achromaticmodel:sec} is extended to accommodate a mass step.
This is accomplished by adding to the predicted magnitude a host-mass dependent term $M_\mathrm{host}$.
The derived parameters
$ \mu[i]; 0 \le i < N-1$ described in  Table~\ref{mu:tab} are now set to   $\sum_{k=0}^{3} \alpha[k] \theta_k[i] + \frac{5}{\log{10}}\left(\frac{\delta z[i+1]}{z[i+1]} -\frac{\delta z[0]}{z[0]}\right) + \left(dm[i+1]-dm[0]\right) + \left(M_\mathrm{host}[i+1]-M_\mathrm{host}[0]\right)$.
Note that while there are $N-1$ magnitude differences, there are $N$ host-mass terms with the inclusion of fiducial SN 0.

The host-mass term for the supernova absolute magnitude is taken to have the form
\begin{equation}
\frac{\Delta_{\mathrm{step}}}{2} \tanh{\left(10(\theta_{M_{\text{host}}}-10)\right)} + M_{\mathrm{low}} + M_{\mathrm{offset}},
\end{equation}
where $\theta_{M_{\text{host}}}$ is the host mass,  $\Delta_{\mathrm{step}}$  is the magnitude difference
between SNe~Ia from extreme low and  high-mass hosts, and the combination $(M_{\mathrm{low}}+M_{\mathrm{offset}})$ is the  
absolute magnitude term appropriate for low-mass hosts.  Hamiltonian
Monte Carlo relies on differentiable posteriors, so the hyperbolic tangent is used to approximate the step function used in other analyses.
The multiplicative factor 10 is large enough to approximate the step while not engendering numerical precision problems.

Supernovae with and without host-mass measurements are treated distinctly.  We would like the measurement
of the mass step to be driven by supernovae with a mass measurement.  Therefore, supernovae
without a mass measurement are assigned a constant magnitude term $M_{\mathrm{offset}}$
not directly coupled to the mass-step parameter.  Thus
\begin{equation}
M_{\mathrm{host}}\mathrm[i]=\begin{cases}
\frac{\Delta_{\mathrm{step}}}{2}
\tanh{\left(10(\theta_{M_{\text{host}}}-10)\right)+ M_{\mathrm{low}} + M_{\mathrm{offset}}}& \quad \text{if SN }i\text{ has mass measurement} \\
M_{\mathrm{offset}} & \quad \text{if SN }i\text{ has no measurement}  \\
\end{cases}.
\end{equation}
Fiducial supernova 0 does have a host measurement.
In the analysis differences between host-mass terms appear
\begin{equation}
M_{\mathrm{host}}\mathrm[i] -M_{\mathrm{host}}\mathrm[0]  = 
\begin{cases}
\frac{\Delta_{\mathrm{step}}}{2} \left(
\tanh{\left(10(\theta_{M_{\text{host}}}[i]-10)\right)} - \tanh{\left(10(\theta_{M_{\text{host}}}[0]-10)\right)} \right)
& \quad \text{if SN }i\text{ has mass measurement} \\
-\frac{\Delta_{\mathrm{step}}}{2}  \tanh{\left(10(\theta_{M_{\text{host}}}[0]-10)\right)} - M_{\mathrm{low}}
 & \quad \text{if SN }i\text{ has no measurement}  \\
\end{cases}.
\end{equation}
The $M_{\mathrm{low}}$ and $\Delta_{\mathrm{step}}$ are the two parameters that describe the relative mass step
relevant in our analysis.

The underlying distribution of host mass is described by a Normal distribution with mean
$\langle M_{\text{host}} \rangle$ and standard deviation
$\sigma_{M_{\text{host}}}$.
The subset of supernovae with a host-galaxy mass measurement has parameters for the true mass $\vec{\theta}_{M_{\text{host}}}$.

The redefined and new parameters 
are summarized in Table~\ref{mass:tab}.

\begin{table}
\begin{center}
\caption{Parameters of the mass-step model.\label{mass:tab}}
\begin{tabular}{rccc}
\hline
Parameter & pdf & Description\\ \hline
$\langle M_{\text{host}} \rangle$ &flat& mass distribution mean \\
$\sigma_{M_{\text{host}}}$ &$\text{Cauchy}(0,2)$; $>0$& mass distribution s.d. \\
$\vec{\theta}_{M_{\text{host}}}$ &$\mathcal{N}(\langle M_{\text{host}} \rangle,\sigma_{M_{\text{host}}})$ &intrinsic host-galaxy mass\\
$m_{\mathrm{hostless}}$ & flat & magnitude of SNe with no host-mass measurement\\
$\Delta_{\mathrm{high}}$ & flat & magnitude offset of high-mass SNe relative to low mass\\
$ \mu[i]; 0 \le i < N-1$ & $\sum_{k=0}^{3} \alpha[k] \theta_k[i] + \frac{5}{\log{10}}\left(\frac{\delta z[i+1]}{z[i+1]} -\frac{\delta z[0]}{z[0]}\right) + \left(dm[i+1]-dm[0]\right) +  \left(M_\mathrm{host}[i+1]-M_\mathrm{host}[0]\right)$ & residual from reference\\
\hline
\end{tabular}
\end{center}
\end{table}

\subsection{Results}

The posteriors of the global parameters are shown in Fig.~\ref{mass_fit:fig}.  The posterior of $\Delta_{\text{high}}$
is within the 2$\sigma$ confidence region.  The step is smaller than the $0.085 \pm 0.028$ mag measured by \citet{2013ApJ...770..108C}. 
The mean and $\pm1\sigma$ mass step is plotted in Fig.~\ref{mass:fig}.  The results for the other parameters are pretty much unchanged from
those shown in \S\ref{achromaticresults:sec}.

\begin{figure}[htbp] %  figure placement: here, top, bottom, or page
   \centering
   \includegraphics[width=2.95in]{../mass_fit.pdf}
   \caption{Posterior contours of the global parameters in the mass-step model.
   \label{mass_fit:fig}}
\end{figure}

The  model host masses are plotted  against the measured input in  Fig.~\ref{masses:fig}.
There is no indication for strong bias between fit and measured host mass.


\begin{figure}[htbp] %  figure placement: here, top, bottom, or page
   \centering
   \includegraphics[width=2.95in]{../masses.pdf}
   \caption{Model fit host-galaxy masses versus input masses.  A line with slope one and zero intercept is plotted for reference.
   \label{masses:fig}}
\end{figure}


\bibliographystyle{apj}
\bibliography{/Users/akim/Documents/alex}


\end{document} 
