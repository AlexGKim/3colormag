\documentclass{aastex61}   	% use "amsart" instead of "article" for AMSLaTeX format
\usepackage{geometry}                		% See geometry.pdf to learn the layout options. There are lots.
\geometry{letterpaper}                   		% ... or a4paper or a5paper or ... 
\usepackage{graphicx}				% Use pdf, png, jpg, or eps§ with pdflatex; use eps in DVI mode
\usepackage{amsmath}
\usepackage{amssymb}
\usepackage{natbib}
\usepackage{lineno}
\usepackage{color}
\defcitealias{1999PASP..111...63F}{F99}
\defcitealias{2017ApJ...842...93M}{M17}
\linenumbers

\begin{document}

\title{}
\author[0000-0001-6315-8743]{A.~G.~Kim}
\affiliation{    Physics Division, Lawrence Berkeley National Laboratory, 
    1 Cyclotron Road, Berkeley, CA, 94720}
%    
%
%\author{     G.~Aldering}
%\affiliation{    Physics Division, Lawrence Berkeley National Laboratory, 
%    1 Cyclotron Road, Berkeley, CA, 94720}
%
%\author{     P.~Antilogus}
%\affiliation{    Laboratoire de Physique Nucl\'eaire et des Hautes \'Energies,
%    Universit\'e Pierre et Marie Curie Paris 6, Universit\'e Paris Diderot Paris 7, CNRS-IN2P3, 
%    4 place Jussieu, 75252 Paris Cedex 05, France}
%    
%\author{     S.~Bailey}
%\affiliation{    Physics Division, Lawrence Berkeley National Laboratory, 
%    1 Cyclotron Road, Berkeley, CA, 94720}
%
%\author{     C.~Baltay}
%\affiliation{    Department of Physics, Yale University, 
%    New Haven, CT, 06250-8121}
%
%\author{     K.~Barbary}
%\affiliation{
%    Department of Physics, University of California Berkeley,
%    366 LeConte Hall MC 7300, Berkeley, CA, 94720-7300}
%
%\author{    D.~Baugh}
%\affiliation{   Tsinghua Center for Astrophysics, Tsinghua University, Beijing 100084, China }
%
%\author{     K.~Boone}
%\affiliation{    Physics Division, Lawrence Berkeley National Laboratory, 
%    1 Cyclotron Road, Berkeley, CA, 94720}
%\affiliation{
%    Department of Physics, University of California Berkeley,
%    366 LeConte Hall MC 7300, Berkeley, CA, 94720-7300}
%
%\author{     S.~Bongard}
%\affiliation{    Laboratoire de Physique Nucl\'eaire et des Hautes \'Energies,
%    Universit\'e Pierre et Marie Curie Paris 6, Universit\'e Paris Diderot Paris 7, CNRS-IN2P3, 
%    4 place Jussieu, 75252 Paris Cedex 05, France}
%
%\author{     C.~Buton}
%\affiliation{    Universit\'e de Lyon, F-69622, Lyon, France ; Universit\'e de Lyon 1, Villeurbanne ; 
%    CNRS/IN2P3, Institut de Physique Nucl\'eaire de Lyon}
%    
%\author{     J.~Chen}
%\affiliation{   Tsinghua Center for Astrophysics, Tsinghua University, Beijing 100084, China }
%
%\author{     N.~Chotard}
%\affiliation{    Universit\'e de Lyon, F-69622, Lyon, France ; Universit\'e de Lyon 1, Villeurbanne ; 
%    CNRS/IN2P3, Institut de Physique Nucl\'eaire de Lyon}
%    
%\author{     Y.~Copin}
%\affiliation{    Universit\'e de Lyon, F-69622, Lyon, France ; Universit\'e de Lyon 1, Villeurbanne ; 
%    CNRS/IN2P3, Institut de Physique Nucl\'eaire de Lyon}
%
%\author{ S.~Dixon}
%\affiliation{
%    Department of Physics, University of California Berkeley,
%    366 LeConte Hall MC 7300, Berkeley, CA, 94720-7300}
%
%\author{     P.~Fagrelius}
%\affiliation{    Physics Division, Lawrence Berkeley National Laboratory, 
%    1 Cyclotron Road, Berkeley, CA, 94720}
%\affiliation{
%    Department of Physics, University of California Berkeley,
%    366 LeConte Hall MC 7300, Berkeley, CA, 94720-7300}
%
%\author{     H.~K.~Fakhouri}
%\affiliation{    Physics Division, Lawrence Berkeley National Laboratory, 
%    1 Cyclotron Road, Berkeley, CA, 94720}
%  \affiliation{
%    Department of Physics, University of California Berkeley,
%    366 LeConte Hall MC 7300, Berkeley, CA, 94720-7300}
%
%\author{     U.~Feindt}
%\affiliation{The Oskar Klein Centre, Department of Physics, AlbaNova, Stockholm University, SE-106 91 Stockholm, Sweden}
%
%\author{     D.~Fouchez}
%\affiliation{    Centre de Physique des Particules de Marseille, 
%    Aix-Marseille Universit\'e , CNRS/IN2P3, 
%    163 avenue de Luminy - Case 902 - 13288 Marseille Cedex 09, France}
%    
%\author{     E.~Gangler}  
%\affiliation{    Clermont Universit\'e, Universit\'e Blaise Pascal, CNRS/IN2P3, Laboratoire de Physique Corpusculaire,
%    BP 10448, F-63000 Clermont-Ferrand, France}
%    
%\author{     B.~Hayden}
%\affiliation{    Physics Division, Lawrence Berkeley National Laboratory, 
%    1 Cyclotron Road, Berkeley, CA, 94720}
%
%\author{     W.~Hillebrandt}
%\affiliation{    Max-Planck-Institut f\"ur Astrophysik, Karl-Schwarzschild-Str. 1,
%D-85748 Garching, Germany}
%
%\author{     M.~Kowalski}
%\affiliation{    Institut fur Physik,  Humboldt-Universitat zu Berlin,
%    Newtonstr. 15, 12489 Berlin}
%\affiliation{ DESY, D-15735 Zeuthen, Germany}
%
%\author{     P.-F.~Leget}
%\affiliation{    Clermont Universit\'e, Universit\'e Blaise Pascal, CNRS/IN2P3, Laboratoire de Physique Corpusculaire,
%    BP 10448, F-63000 Clermont-Ferrand, France}
%    
%\author{     S.~Lombardo}
%\affiliation{    Institut fur Physik,  Humboldt-Universitat zu Berlin,
%    Newtonstr. 15, 12489 Berlin}
%    
%\author{     J.~Nordin}
%\affiliation{    Institut fur Physik,  Humboldt-Universitat zu Berlin,
%    Newtonstr. 15, 12489 Berlin}
%    
%\author{     R.~Pain}
%\affiliation{    Laboratoire de Physique Nucl\'eaire et des Hautes \'Energies,
%    Universit\'e Pierre et Marie Curie Paris 6, Universit\'e Paris Diderot Paris 7, CNRS-IN2P3, 
%    4 place Jussieu, 75252 Paris Cedex 05, France}
%     
%\author{     E.~Pecontal}
%\affiliation{   Centre de Recherche Astronomique de Lyon, Universit\'e Lyon 1,
%    9 Avenue Charles Andr\'e, 69561 Saint Genis Laval Cedex, France}
%    
%\author{    R.~Pereira}
% \affiliation{    Universit\'e de Lyon, F-69622, Lyon, France ; Universit\'e de Lyon 1, Villeurbanne ; 
%    CNRS/IN2P3, Institut de Physique Nucl\'eaire de Lyon}
% 
% \author{    S.~Perlmutter}
% \affiliation{    Physics Division, Lawrence Berkeley National Laboratory, 
%    1 Cyclotron Road, Berkeley, CA, 94720} 
%\affiliation{
%    Department of Physics, University of California Berkeley,
%    366 LeConte Hall MC 7300, Berkeley, CA, 94720-7300}
%    
% \author{    D.~Rabinowitz}
% \affiliation{    Department of Physics, Yale University, 
%    New Haven, CT, 06250-8121}
%    
% \author{    M.~Rigault} 
%\affiliation{    Institut fur Physik,  Humboldt-Universitat zu Berlin,
%    Newtonstr. 15, 12489 Berlin}
%     
% \author{    D.~Rubin}
% \affiliation{    Physics Division, Lawrence Berkeley National Laboratory, 
%    1 Cyclotron Road, Berkeley, CA, 94720}
%    \affiliation{   Space Telescope Science Institute, 3700 San Martin Drive, Baltimore, MD 21218}
% 
% \author{    K.~Runge}
% \affiliation{    Physics Division, Lawrence Berkeley National Laboratory, 
%    1 Cyclotron Road, Berkeley, CA, 94720}
% 
% \author{    C.~Saunders}
% \affiliation{    Physics Division, Lawrence Berkeley National Laboratory, 
%    1 Cyclotron Road, Berkeley, CA, 94720}
%
%\author{    C.~Sofiatti}
%\affiliation{    Physics Division, Lawrence Berkeley National Laboratory, 
%    1 Cyclotron Road, Berkeley, CA, 94720} 
%\affiliation{
%    Department of Physics, University of California Berkeley,
%    366 LeConte Hall MC 7300, Berkeley, CA, 94720-7300}
%
%\author{    N.~Suzuki}
%\affiliation{    Physics Division, Lawrence Berkeley National Laboratory, 
%    1 Cyclotron Road, Berkeley, CA, 94720}
%
%\author{     S.~Taubenberger}
%\affiliation{    Max-Planck-Institut f\"ur Astrophysik, Karl-Schwarzschild-Str. 1,
%D-85748 Garching, Germany}
%
%\author{     C.~Tao}
%\affiliation{   Tsinghua Center for Astrophysics, Tsinghua University, Beijing 100084, China }
%\affiliation{    Centre de Physique des Particules de Marseille, 
%    Aix-Marseille Universit\'e , CNRS/IN2P3, 
%    163 avenue de Luminy - Case 902 - 13288 Marseille Cedex 09, France}
%   
%\author{     R.~C.~Thomas}
%\affiliation{    Computational Cosmology Center, Computational Research Division, Lawrence Berkeley National Laboratory, 
%    1 Cyclotron Road MS 50B-4206, Berkeley, CA, 94720}
%    
%\collaboration{(The Nearby Supernova Factory)}


%\begin{abstract}
%\end{abstract}
%
%\keywords{supernovae: general; supernovae}

\section{Introduction}
The objective of this work is to determine the relative absolute magnitudes of SNe~Ia.  This is done in two steps.
The first step is to model peak-magnitude colors as a function of parameters.  The second step is to model the relative achromatic magnitudes
as a function of the same parameters.
The two models in combination give the  relative absolute magnitudes of SNe~Ia.
The color analysis is expounded upon in the earlier draft article.  This note focuses on the relative achromatic magnitudes.

\section{Color Analysis Summary}
The color analysis gives per-supernova parameters
$\vec{\Delta}$,  $\overrightarrow{EW}_{Ca}$,  $\overrightarrow{EW}_{Si}$,  $\vec{\lambda}_{Si}$,   $\vec{x}_1$,  $\sigma_p\phi_{\hat{V}}  \vec{p}$,
where each vector has $N$ elements, one for each of the supernovae in the sample.  The parameters $g_0$ and $g_1$ are attributed to dust
and are not included as per-supernova parameters.  The CMB redshifts of the supernovae are $\vec{z}$.  Define a new set of vectors such that
$\vec{\Delta}_{.0} = (\vec{\Delta} - \Delta[0])[1:]$ with length $N-1$; here each per-supernova parameter is measured relative to the
reference supernova arbitrarily selected to be the one with index 0.

The data vector for the current analysis is the concatenation of these vectors
\[
\vec{y} = \left\{\vec{\Delta}_{.0} , \overrightarrow{EW}_{Ca\,.0} ,  \overrightarrow{EW}_{Si\,.0},  \vec{\lambda}_{Si\,.0},   \vec{x}_{1\,.0},  \sigma_p\phi_{\hat{V}}  \vec{p}_{.0} \right\},
\]
which has length $6(N-1)$.  The data covariance $C$ is given by the color analysis.\footnote{In
the color analysis the per-SN parameters
are not independent.  The parameters relative to an arbitrary reference supernova are independent.}
The standard deviations of $\vec{y}$ excepting the first is denoted $\vec{\sigma}_{y_{.0}})$.

The model has the following fundamental parameters that describe the underlying distribution of SN feature parameters.
This top level of the hierarchical model is necessary to avoid Eddington bias and simultaneously constrain the multiple linear
coefficients in our model for intrinsic grey-magnitude residuals.
The population distribution parameters are
\begin{center}
\begin{tabular}{rccc}
\hline
Parameter & pdf & Description & \# parameters\\ \hline
$\vec{\theta}_{\mathit{mn}}$ & flat  &  mean & 5\\
$\vec{\sigma}_{\theta}$ & $\text{Cauchy}(\vec{0},4\vec{\sigma}_{y_{.0}})$; $>0$  &  s.d. & 5\\
$C_{\theta}$ & $\text{LKJ}(4)$ &  correlation & 10\\
\hline
\end{tabular}
\end{center}
This amounts to $20$ parameters.

\section{Achromatic Magnitude Analysis}

The model has the following fundamental parameters:
\begin{center}
\begin{tabular}{rccc}
\hline
Parameter & pdf & Description & \# parameters\\ \hline
$\sigma_M$ & $\text{Cauchy}(0.08, 0.08)$; $>0$ & intrinsic dispersion & 1\\
$\vec{\theta}_{EW_{Ca}}$ & Eq.~\ref{feature:pdf} & ${EW}_{Ca}$ features & $N-1$ \\
$\vec{\theta}_{EW_{Si}}$ & Eq.~\ref{feature:pdf}& ${EW}_{Si}$ features & $N-1$ \\
$\vec{\theta}_{\lambda_{Si}}$ &Eq.~\ref{feature:pdf}& ${\lambda}_{Si}$ features & $N-1$ \\
$\vec{\theta}_{x_1}$ & Eq.~\ref{feature:pdf}& ${x}_{1}$ features & $N-1$ \\
$\vec{\theta}_{p}$ & Eq.~\ref{feature:pdf}& $\sigma_p\phi_{\hat{V}}  {p} $ features & $N-1$ \\
$\vec{\alpha}$ & Flat& feature cofactors & 5 \\
\hline
\end{tabular}
\end{center}
The feature pdf's are independent for each supernova, such that SN indexed by $i$ is associated with the parameters
$\vec{\theta}^{(i)}=\left\{\theta_{EW_{Ca}}[i], \theta_{EW_{Si}}[i], \theta_{\lambda_{Si}}[i], \theta_{x_1}[i], \theta_{p}[i] \right\}$
that are drawn as
\begin{align}
\vec{\theta}^{(i)}_0 & \sim \mathcal{N}(\vec{\theta}_{\mathit{mn}}, \vec{\theta}_{\sigma})  \nonumber \\
\vec{\theta}^{(i)} & \sim \mathcal{N}(\vec{\theta}^{(i)}_0, C_{\theta}) .
\label{feature:pdf}
\end{align}
Visual inspection of the input feature distributions finds that they are not grossly multimodal.
These are $5N+1$ of these parameters.

For expository and computational convenience the following parameters are introduced:  
\begin{center}
\begin{tabular}{rccc}
\hline
Parameter & pdf & Description & \# parameters\\ \hline
$\vec{v}$ & $\mathcal{N}(0,10^{-3})$ &  peculiar velocity & N \\\
$ \overrightarrow{dm}$ &  $\mathcal{N}(0,\sigma_M)$  &  peculiar magnitude & N \\
\hline
\end{tabular}
\end{center}
These parameters are not uniquely constrained but manifest dispersion.

The following derived parameters are defined from the above parameters:
\begin{center}
\begin{tabular}{rccc}
\hline
Parameter & Definition & Description\\ \hline
$\overrightarrow{\delta z}$ & $\sqrt{(1+\vec{v})/ (1-\vec{v})}-1$ & peculiar redshift\\
$ \mu[i]; 0 \le i < N-1$ & $\sum_{k=0}^{4} \alpha[k] \theta_k[i] + \frac{5}{\log{10}}\left(\frac{\delta z[i+1]}{z[i+1]} -\frac{\delta z[0]}{z[0]}\right) + \left(dm[i+1]-dm[0]\right)$ & residual from reference\\
$ \mu[N:6(N-1)]$ & $\left\{\vec{\theta}_{EW_{Ca}},\vec{\theta}_{EW_{Si}}, \vec{\theta}_{\lambda_{Si}}, \vec{\theta}_{x_1},\vec{\theta}_{p} \right\}$ &  features \\
\hline
\end{tabular}
\end{center}
The residual relative to reference SN~0 observed as $\vec{\Delta}_{.0}$
is modeled as a linear function of the supernova features plus terms for peculiar velocity and intrinsic dispersion.

The model likelihood is described by
\begin{equation*}
\vec{y} \sim \mathcal{N}\left(\vec{\mu}, C\right).
\end{equation*}
%\bibliographystyle{apj}
%\bibliography{/Users/akim/Documents/alex}


\end{document} 
